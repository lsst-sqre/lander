\section{Introduction}
The Large Synoptic Survey Telescope (\gls{LSST}) will usher in a new era of data-intensive astronomy. The observing program will observe the southern sky repeatedly over 10 years in 6 bands providing an unprecedented census of the astrophysical bodies in the universe.  Funded by the \gls{NSF} and the \gls{DOE}, this keystone observatory is due to go into operations in October 2022.

Producing 20TB of data a night, this is a huge step up in data acquisition from other optical telescopes. At its conception this was considered an ominous data volume requiring highly specialized computing infrastructure. In the intervening time, however, the growth of planetary-scale industry services (such us Goggle or Facebook) has resulted in software, engineering techniques and infrastructure that render this sort of data flow routine. \gls{LSST}  operations are expected to cost tens of millions a year with order of \$10M computing budget.

The \gls{LSST} computing load is a poster child for cloud computing - the science platform is designed for kubernetes and the
data release processing fits perfectly with opportunistic compute pricing. This is a large \gls{NSF} project and the the one most suited and ready for cloud deployment.
We have used some pathfinder deployments to demonstrate that it is feasible to use commercial cloud providers for the \gls{LSST} \gls{Data Management} production system. Such a move would bring significant technological and operational advantages; the barrier to acting on this is price and uncertainty on future pricing.

A solution might be to reach a fixed price partnership for a cloud-based deployment of the \gls{Data Management} systems in which Google undertake to provide \emph{do what is needed} for success at some reasonable and agreed-upon annual fee.

\section{Studies to date}
We have demonstrated some of the major components of \gls{LSST} \gls{Data Management} work well on Google  \citedsp{DMTN-072}.

We deployed \gls{Qserv} on Google with reasonable performance.
We demonstrated Data transfer  could adequate for \gls{Prompt Processing}, within the limits of the available network.
The Prompt Product Database was  stood up and tested.  The  \gls{Science Platform} was deployed and users simulated.
The later of course is designed around kubernetes and made for Google Cloud.

A detailed report may be found in \citeds{DMTN-125}.

We have not considered bulk transfer of data to other partners - Google may be much better placed to do this than any computing centre.

\section{LSST compute and storage needs}

The greatest cost driver is storage - we accumulate about 50PB a year of data. All of this needs to be processed annually. Hence in year 10 we need to access about half an Exabyte of data. Not all of this will be regularly accessed, it is likely few of the raw images will be reprocessed by individual astronomers.

\tabref{tab:Inputs} gives a rough overview of compute and storage needs.

\tiny \begin{longtable} { |p{0.22\textwidth}  |r  |r  |r  |r  |r |}
\caption{Various inputs for deriving costs \label{tab:Inputs}}\\
\hline
\textbf{Year}&\textbf{2019}&\textbf{2020}&\textbf{2021}&\textbf{2022}&\textbf{2023} \\ \hline
{FLOPs Needed Total (no Alerts)}&{1.00E+19}&{9.48261E+19}&{1.00E+19}&{4.74131E+20}&{5.91525E+20} \\ \hline
{Time to Process days}&{252.0}&{252.0}&{252.0}&{252.0}&{252.0} \\ \hline
{Time to Process seconds}&{21772800.0}&{21772800.0}&{21772800.0}&{21772800.0}&{21772800.0} \\ \hline
{Instantaneous GFLOP/ s}&{4.59E+02}&{4355.255691}&{4.59E+02}&{21776.27846}&{27168.07327} \\ \hline
{Instantaneous GFLOP/ s (inc Alerts)}&{4.59E+02}&{30025.25569}&{2.61E+04}&{21776.27846}&{27168.07327} \\ \hline
{Disk Space TB}&{5000}&{10000}&{20000}&{50000}&{100000} \\ \hline
{I/ O for year TB}&{15000}&{30000}&{60000}&{150000}&{300000} \\ \hline
{Base numbers }&{GFLOP}&&&& \\ \hline
{LDM-138 DR1,2 Data Rel sheet row 1}&{426717500000}&{}&{97381399021}&& \\ \hline
{LDM-138 DR3 Data Rel sheet row 2}&{959090000000}&&&& \\ \hline
{LDM-138 Alert Instananeous}&{25670}&&&& \\ \hline
{Alert Total, assuming 275k visits/ year}&{177219625000}&&&& \\ \hline
\textbf{Total Yr1 (inc DAC)}&\textbf{474130555556}&&&& \\ \hline
\end{longtable} \normalsize



\section{Cost}
Currently the cloud computing cost models do  not align well with federal research computing plans. We believe this gap will narrow or disappear in five years or so, however in the meantime LSST and Google may miss an opportunity to move a major research project on to commodity cloud computing.

Using the information from the Google study where we ran some of our real processes, we have come to a price for running the \gls{Science Platform} and storing data on Google. For 15PB of storage and a modest K8S cluster to host the platform the projected 2022 cost is around \$3M of which more than $\frac{2}{3}$ are storage costs.

Though a good price this is not a sustainable price for \gls{LSST}, we can construct petascale storage we would own and use for 5 years for under \$200K a petabyte (the implied \emph{annual} price at google).  We require about 50 Petabytes a year for 10 years. The out year costs look prohibitive on the cloud.

The cost of compute is probably not an issue in comparison - we can use spot/interuptable instance pricing for \gls{DRP}.

%Based on the proof of concept some prices were calculated in \tabref{tab:Google}.
%\tiny \begin{longtable} { |p{0.22\textwidth}  |r  |r  |r |}
\caption{Price estimates from google POC \label{tab:Google}}\\
\hline
{Google Compute}&{per month}&{1 year price}&{Price GFLOP} \\ \hline
{POC price GFLOP (n1-highmen-4)}&{\$242.00}&{\$2,904.00}&{\$2.60} \\ \hline
{LIkely (inefficeny included)}&{}&{}&{\$5.34} \\ \hline
{Pessimistic (double that)}&{}&{}&{\$10.69} \\ \hline
{Google Storage}&{GB/ m}&{TB/ month}&{note } \\ \hline
{Optm. }&{0.007}&{\$84.00}& \\ \hline
{Likely (HDD)}&{0.03999975641}&{\$480.00}&{26cent on web} \\ \hline
{Pessimiistic (SSD)}&{0.1699908088}&{\$2,039.89}& \\ \hline
\end{longtable} \normalsize




\section{Conclusion}

LSST \gls{DM} has been constructing a cloud ready system for many years. We believe commercial cloud is the correct approach but we may be a few years ahead of commercial and federal cost models aligning. We hope that we may be able to partner with google to usher in a new ear of federally funded research in the cloud.

~

